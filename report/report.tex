\documentclass{article}

\usepackage{arxiv}

\usepackage[utf8]{inputenc} % allow utf-8 input
\usepackage[T1]{fontenc}    % use 8-bit T1 fonts
\usepackage{lmodern}        % https://github.com/rstudio/rticles/issues/343
\usepackage{hyperref}       % hyperlinks
\usepackage{url}            % simple URL typesetting
\usepackage{booktabs}       % professional-quality tables
\usepackage{amsfonts}       % blackboard math symbols
\usepackage{nicefrac}       % compact symbols for 1/2, etc.
\usepackage{microtype}      % microtypography
\usepackage{lipsum}
\usepackage{graphicx}

\title{Longitudinal Data Cleaning - A Case of the NLSY79 Data}

\author{
    Dewi Amaliah
    \thanks{Use footnote for providing further information about author (webpage,
alternative address)---\emph{not} for acknowledging funding agencies.
Optional.}
   \\
    Department of Econometric and Business Statistics \\
    Monash University \\
  Clayton, VIC 3168 \\
  \texttt{\href{mailto:dama0007@student.monash.edu}{\nolinkurl{dama0007@student.monash.edu}}} \\
   \And
    Dianne Cook
   \\
    Department of Econometric and Business Statistics \\
    Monash University \\
  Clayton, VIC 3168 \\
  \texttt{\href{mailto:dicook@monash.edu}{\nolinkurl{dicook@monash.edu}}} \\
  }


% Pandoc citation processing



\begin{document}
\maketitle

\def\tightlist{}


\begin{abstract}
Enter the text of your abstract here.
\end{abstract}

\keywords{
    longitudinal data
   \and
    data cleaning
   \and
    data tidying
   \and
    robust linear model
  }

\hypertarget{introduction}{%
\section{Introduction}\label{introduction}}

\hypertarget{the-nlsy79}{%
\section{The NLSY79}\label{the-nlsy79}}

\hypertarget{the-nlsy79-data-cleaning}{%
\section{The NLSY79 Data Cleaning}\label{the-nlsy79-data-cleaning}}

\hypertarget{getting-and-tyding-the-data}{%
\subsection{Getting and Tyding the
Data}\label{getting-and-tyding-the-data}}

The NLYS data is stored in a
\href{https://www.nlsinfo.org/content/cohorts/nlsy79/get-data}{database}
and could be downloaded by variables. Several variables are available to
be downloaded and could be browsed by index. For the wages data set, we
only extracted these variables:

\begin{itemize}
\tightlist
\item
  Education, Training \& Achievement Scores

  \begin{itemize}
  \tightlist
  \item
    Education -\textgreater{} Summary measures -\textgreater{} All
    schools -\textgreater{} By year -\textgreater{} Highest grade
    completed

    \begin{itemize}
    \tightlist
    \item
      Downloaded all of the 78 variables in Highest grade completed.
    \end{itemize}
  \end{itemize}
\item
  Employment

  \begin{itemize}
  \tightlist
  \item
    Summary measures -\textgreater{} By job -\textgreater{} Hours worked
    and Hourly wages

    \begin{itemize}
    \tightlist
    \item
      Downloaded all of the 427 variables in Hours worked
    \item
      Downloaded all of the 151 variables in Hourly wages
    \end{itemize}

    Both hours worked and hourly wages are recorded by the job, up to
    five jobs for each id/subject.
  \end{itemize}
\item
  Household, Geography \& Contextual Variables

  \begin{itemize}
  \tightlist
  \item
    Context -\textgreater{} Summary measures -\textgreater{} Basic
    demographics

    \begin{itemize}
    \tightlist
    \item
      Downloaded year and month of birth, race, and sex variables.
    \end{itemize}

    There are two versions of the year and month of birth, i.e.~1979 and
    1981 data. We downloaded these two versions.
  \end{itemize}
\end{itemize}

The downloaded data set came in a csv (wages-high-school-demo.csv) and
dat (wages-high-school-demo.dat) format. We only used the .dat format.
It also came along with these files:

\begin{itemize}
\tightlist
\item
  wages-high-school-demo.NLSY79: This is the tagset of variables that
  can be uploaded to the web site to recreate the data set.
\item
  wages-high-school-demo.R: This is an R script provided automatically
  by the database for reading the data into R and convert the variables'
  name and its label into something more sensible. We utilized this code
  to do an initial data tidying. It produced two data set,
  \texttt{categories\_qnames} (the observations are stored in
  categorical/interval values) and \texttt{new\_data\_qnames} (the
  observations are stored in integer form). In this case, we only used
  the latter.
\end{itemize}

\texttt{new\_data\_qnames} is still untidy, where all of the variables
for each year and each job being stored in one column, hence the data
contains a huge number of columns (686 columns). Thus, the data should
be tidied and wrangled first to extract the demographic and employment
variables that we want to put in the final data set.

\hypertarget{tidying-demographic-variables}{%
\subsubsection{Tidying demographic
variables}\label{tidying-demographic-variables}}

Date of birth, age in 1979, gender, race, highest grade completed
(factor and integer), and the year when the highest grade completed are
the variables that we want to use in the cleaned data set. There are two
versions of date of birth variable, which are the 1979 version and the
1981 version. In this case, we only used the 1979 data. We also did a
consistency check for the 1979 and 1981 data and flag the inconsistent
observations.

\hypertarget{initial-data-analysis}{%
\subsection{Initial Data Analysis}\label{initial-data-analysis}}

\hypertarget{robust-linear-model-for-influential-values-treatment}{%
\subsection{Robust Linear Model for Influential Values
Treatment}\label{robust-linear-model-for-influential-values-treatment}}

\hypertarget{conclusion}{%
\section{Conclusion}\label{conclusion}}

\bibliographystyle{unsrt}
\bibliography{references.bib}


\end{document}
